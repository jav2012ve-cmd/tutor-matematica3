
\documentclass{article}
\usepackage[utf8]{inputenc}
\usepackage{amsmath}
\usepackage{amssymb}
\begin{document}
\begin{center}
    Universidad Católica Andrés Bello \\
    Escuela de Economía - Matemáticas III \\
    \textbf{CONSTANCIA DE AUTOEVALUACIÓN}
\end{center}

\vspace{0.5cm}
\noindent \textbf{Calificación:} 5.0/20 pts \\
\vspace{0.5cm}

\section*{Detalle de la Prueba}
\begin{enumerate}
\item \textbf{Pregunta:} Calcule la siguiente integral indefinida, prestando atención al diferencial:
$$ \int (4x^3 + 2k^2) dx $$
(Considere $k$ como una constante).
\begin{itemize}
\item \textit{Respuesta del Estudiante:} A) $x^4 + 2k^2x + C$
\item \textit{Respuesta Correcta:} A) $x^4 + 2k^2x + C$
\item \textit{Puntos:} 2.5
\end{itemize}

\item \textbf{Pregunta:} Dada la integral $\int x^2 e^{x^3+1} dx$, aplique el cambio de variable $u = x^3+1$. ¿Cuál es la integral resultante en términos de $u$?
\begin{itemize}
\item \textit{Respuesta del Estudiante:} B) $\int e^u du$
\item \textit{Respuesta Correcta:} A) $\frac{1}{3} \int e^u du$
\item \textit{Puntos:} 0
\end{itemize}

\item \textbf{Pregunta:} Para resolver la integral $\int \frac{x^3+2x^2-1}{x^2+x} dx$, ¿cuál es la forma correcta de la expresión después de realizar la división de polinomios y antes de aplicar fracciones simples?
\begin{itemize}
\item \textit{Respuesta del Estudiante:} C) $\int \left(x+2 + \frac{-2x-1}{x^2+x}\right) dx$
\item \textit{Respuesta Correcta:} A) $\int \left(x+1 + \frac{-x-1}{x^2+x}\right) dx$
\item \textit{Puntos:} 0
\end{itemize}

\item \textbf{Pregunta:} Calcule la siguiente integral indefinida:
$$ \int x \ln(x) dx $$
\begin{itemize}
\item \textit{Respuesta del Estudiante:} C) $\frac{x^2}{2} \ln(x) - \frac{x^2}{2} + C$
\item \textit{Respuesta Correcta:} A) $\frac{x^2}{2} \ln(x) - \frac{x^2}{4} + C$
\item \textit{Puntos:} 0
\end{itemize}

\item \textbf{Pregunta:} Plantee la integral o suma de integrales necesarias para calcular el área encerrada por las funciones $f(x) = x^2 - 2x$ y $g(x) = x$.
\begin{itemize}
\item \textit{Respuesta del Estudiante:} C) $\int_{0}^{2} (x - (x^2 - 2x)) dx + \int_{2}^{3} ((x^2 - 2x) - x) dx$
\item \textit{Respuesta Correcta:} A) $\int_{0}^{3} (x - (x^2 - 2x)) dx$
\item \textit{Puntos:} 0
\end{itemize}

\item \textbf{Pregunta:} Dadas las funciones de demanda $P_D(Q) = 10 - Q$ y de oferta $P_O(Q) = 2 + Q$. Calcule el Excedente del Consumidor (EC).
\begin{itemize}
\item \textit{Respuesta del Estudiante:} A) $EC = 16$
\item \textit{Respuesta Correcta:} C) $EC = 8$
\item \textit{Puntos:} 0
\end{itemize}

\item \textbf{Pregunta:} Calcule la siguiente integral indefinida:
$$ \int \frac{1}{x^2-9} dx $$
\begin{itemize}
\item \textit{Respuesta del Estudiante:} C) $\frac{1}{6} \ln\left|x^2-9\right| + C$
\item \textit{Respuesta Correcta:} A) $\frac{1}{6} \ln\left|\frac{x-3}{x+3}\right| + C$
\item \textit{Puntos:} 0
\end{itemize}

\item \textbf{Pregunta:} Al aplicar la técnica de integración por partes $\int u dv = uv - \int v du$, ¿cuál de las siguientes afirmaciones es generalmente la más útil para elegir $u$ y $dv$?
\begin{itemize}
\item \textit{Respuesta del Estudiante:} B) Elegir $u$ como la parte que se simplifica al derivar y $dv$ como la parte fácil de integrar.
\item \textit{Respuesta Correcta:} B) Elegir $u$ como la parte que se simplifica al derivar y $dv$ como la parte fácil de integrar.
\item \textit{Puntos:} 2.5
\end{itemize}


\end{enumerate}
\end{document}
